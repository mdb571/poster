%
% these are some of my own latex definitions
%

\newcommand{\lzhu}{\underline{L.~Zhu}}
\renewcommand{\topfraction}{.75}

%%%% for manuscript submission
%\newenvironment{annotation}{\bfseries}{\normalfont}
\newenvironment{annotation}{\color{red}}{\color{black}}
%\newcommand{\clearemptydoublepage}{\newpage{\pagestyle{empty}\cleardoublepage}}

%%% for inserting figures
\newcommand{\putfigure}[2]{%
  \begin{center}
%    \epsfig{file=#2,width=#1}
     \includegraphics[width=#1]{#2}%
  \end{center}
}

% two figs left and right aligned at the bottom
\newcommand{\putfigureLR}[4]{%
  \begin{center}
     \begin{tabular}{cc}
     \includegraphics[width=#1]{#2} &
     \includegraphics[width=#3]{#4}
     \end{tabular}
  \end{center}
}

% two figs left and right aligned at the bottom
\newcommand{\putfigurecaptionLR}[6]{%
    \hspace{2em}
    \begin{minipage}{#1}
    %\centering
    \footnotesize
    \textbf{#2}
    \itshape
    #3
    \end{minipage}
    \hspace{1em}
    \begin{minipage}{#4}
    % \centering
    \footnotesize
    \textbf{#5}
    \itshape
    #6
    \end{minipage}

}


% four figs left and right
\newcommand{\putfigureLRLR}[5]{%
  \begin{center}
     \begin{tabular}{cc}
     \includegraphics[width=#1]{#2} &
     \includegraphics[width=#1]{#3} \\
     \includegraphics[width=#1]{#4} &
     \includegraphics[width=#1]{#5}
     \end{tabular}
  \end{center}
}

% two figs left and right aligned at the top
\def\imagetop#1{\vtop{\null\hbox{#1}}}
\newcommand{\putfigureLRt}[4]{%
  \begin{center}
     \begin{tabular}{cc}
     \imagetop{\includegraphics[width=#1]{#2}} &
     \imagetop{\includegraphics[width=#3]{#4}}
     \end{tabular}
  \end{center}
}

% three figs in a row
\newcommand{\putfigureLCR}[6]{%
  \begin{center}
     \begin{tabular}{ccc}
     \includegraphics[width=#1]{#2} &
     \includegraphics[width=#3]{#4} &
     \includegraphics[width=#5]{#6}
     \end{tabular}
  \end{center}
}

% three figs in a row aligned at the top
\newcommand{\putfigureLCRt}[6]{%
  \begin{center}
     \begin{tabular}{ccc}
     \imagetop{\includegraphics[width=#1]{#2}} &
     \imagetop{\includegraphics[width=#3]{#4}} &
     \imagetop{\includegraphics[width=#5]{#6}}
     \end{tabular}
  \end{center}
}

% one left and two in a column at right
\newcommand{\putfigureLRR}[5]{%
  \begin{center}
     \begin{tabular}{cc}
     \includegraphics[width=#1]{#2} &
     \parbox[b]{#3}{\includegraphics[width=#3]{#4}
     \includegraphics[width=#3]{#5}}
     \end{tabular}
  \end{center}
}

% two left in a column and one at right
\newcommand{\putfigureLLR}[5]{%
  \begin{center}
     \begin{tabular}{cc}
     \parbox[b]{#1}{\includegraphics[width=#1]{#2}\\
     \includegraphics[width=#1]{#3}} &
     \includegraphics[width=#4]{#5}
     \end{tabular}
  \end{center}
}

% one fig at left and text in the right
\newcommand{\putfigureL}[4]{%
  \begin{center}
  \parbox{#1}{\includegraphics[width=#1]{#2}}
  \parbox{#3}{#4}
  \end{center}
}

% two figs at left and text in the right
\newcommand{\putfigureLL}[5]{%
  \begin{center}
  \parbox{#1}{\includegraphics[width=#1]{#2}
              \includegraphics[width=#1]{#3}}
  \parbox{#4}{#5}
  \end{center}
}

% two columns of figs
\newcommand{\putfigureLLRR}[6]{%
  \begin{center}
  \parbox{#1}{\includegraphics[width=#1]{#2}
              \includegraphics[width=#1]{#3}}
  \parbox{#4}{\includegraphics[width=#4]{#5}
              \includegraphics[width=#4]{#6}}
  \end{center}
}

%%% two columns texts
\newcommand{\twocol}[4]{%
  \begin{center}
  \parbox[t][][t]{#1}{#2}
  \parbox[t][][t]{#3}{#4}
  \end{center}
}

%%% slide heading
\newcommand{\slideheading}[1]{%
  \begin{center}
    \large\bf
    \shadowbox{#1}%
  \end{center}
  \vspace{1ex minus 1ex}}

%%% some often used units
\newcommand{\kms}{km/s}
\renewcommand{\deg}{$^\circ$}
\newcommand{\Cel}{{\deg}C}

%%% math symbols: e, d (derivative), i (imaginary), T (transpose)
\newcommand{\me}{\mathrm{e}}
\newcommand{\md}{\mathrm{d}}
\newcommand{\mi}{\mathrm{i}}
\newcommand{\mT}{\mathrm{T}}

%%% differential operator
\newcommand{\dif}[2]{\frac{\md #1}{\md #2}}
\newcommand{\ddif}[2]{\frac{\md^2 #1}{\md #2^2}}
\newcommand{\dd}[1]{#1^{\prime\prime}}
\newcommand{\ddd}[1]{#1^{\prime\prime\prime}}
\newcommand{\pdif}[2]{\frac{\partial #1}{\partial #2}}
\newcommand{\pdifc}[2]{{#1}_{,#2}}
\newcommand{\pddif}[2]{\frac{\partial^2 #1}{\partial #2^2}}
\newcommand{\pdddif}[3]{\frac{\partial^2 #1}{\partial #2\partial #3}}
\newcommand{\grad}{\nabla}
\newcommand{\rgrad}{\overrightarrow{\grad}}
\newcommand{\lgrad}{\overleftarrow{\grad}}
\renewcommand{\div}{\nabla\cdot}
\newcommand{\curl}{\grad \times}
\newcommand{\laplace}{\grad^2}
\newcommand{\sgn}{\operatorname{sgn}}

\providecommand{\conjg}[1]{#1^{*}}
\providecommand{\abs}[1]{\lvert#1\rvert}
\providecommand{\norm}[1]{\lVert#1\rVert}

%%%  vectors
\newcommand{\vct}[1]{\mathbf{#1}}
\newcommand{\uct}[1]{\hat{\mathrm{#1}}}

%%% matrices
\newcommand{\mtx}[1]{\mathbf{#1}}

%%% tensors
\newcommand{\tensor}[1]{\mathbf{#1}}

%%% isotopes
\newcommand{\isotope}[2]{{}^{#1}#2}

\def\cossin{\begin{pmatrix}\cos{n \theta}\\ \sin{n \theta}\end{pmatrix}}
\def\sincos{\begin{pmatrix}\sin{n \theta}\\ \cos{n \theta}\end{pmatrix}}

%\DeclareMathOperator\erf{erf}
%\DeclareMathOperator\erfc{erfc}
\endinput
